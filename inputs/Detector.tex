\chapter{The LHC-ATLAS Experiment}
\section{Large Hadron Collider}

The Large Hadron Collider (LHC) is the circular collider which is located at CERN, Geneva. It was built inside the existing 27~km tunnel, previously housing the Large Electron-Positron Collider (LEPP). The LHC project was planned in the 1990s and its construction started in 2000. \\
The LHC accelerate mainly protons and also heavy ions such as lead (Pb) in two beams running in opposite directions and collide them. 
The machine has been operating since 2008 and the first physics run, referred as Run~1 was completed in 2010 to 2012, with the center-of-mass energy of $\sqrt{s}$ = 7 to 8 TeV. The Higgs boson was discovered by ATLAS and CMS experiment in 2012 \ref{}, which is one of the most important accomplishments of this period. The energy was increased to $\sqrt{s}$ = 13TeV in 2015 to 2018, this was the second phase of the LHC, referred as Run~2. The primary design energy of the accelerator is $\sqrt{s}$ = 14TeV, which is close the current energy.
The higher amount of data taken from the LHC and its increasing in energy allowed to pushed the limit of our understanding of the particle physics. 
\subsection{Accelerator complex}
%pre accelerator
The particles are gradually accelerated by a chain of linear and circular accelerator complex, shown in Figure~\ref{fig:accelerator}. At first protons were in the form of hydrogen atoms. After being ionized, they are fed in the the linear accelerator (LINAC), where they reach to energy of 50 MeV and formed into bunches. Then they are guided to the Proton Synchrotoron Booster (PSB). The PSB consists of four synchrotorons of 25m which bring protons' energy up to 1.4~GeV. The proton enter the Proton Synchrotron (PS) after that. This PS is a 638~m synchrotoron and it increases the energy of the protons up to 25~GeV. The final injector for the LHC is Super Proton Synchrotoron (SPS), increasing the protons energy up to 450~GeV with 7~km radius synchrotoron. The SPS also provides particles for some CERN based experiments such as AWAKE, as well as the CERN testbeam areas. \\
\begin{figure}[tbp]
\begin{center}
%\subfigure[]{
 \includegraphics[width=1.0\textwidth,keepaspectratio]{figures/detector/CERN}
%}
\caption{
The whole picture of the CERN accelerator complex\ref{}
}
\label{fig:accelerator}
\end{center}
\end{figure}
%electric and magnetic field
The SPS injects protons to the LHC with the enegy of 450~GeV in the form of bunches, separated by 25~ns and each contains 1.15 $\times$ 10$^{11}$ protons. They are circulated in the two separated beam pipes. They are accelerated by and electric field while being kept curved by a magnetic field. The accelerating is performed by the 8 radio frequency cavities.They provides a frequency of 400~MHz and generate the electric fields. They are superconducting and operated at a temperature of 4.5~K. The beams are bent by the 1232 super-conducting dipole magnets to be kept in the circle trajectory. The magnetic field strength is 8.3~T, and they are operated at a temperature of 1.9~K. In addition 392 quadrupole magnets are used to focus the beams. \\
%The 4 experiments
When they reached to the desired energy, they are brought and collided in four experiments, in four interaction points (IP). 
The experiments are ATLAS\ref{}, CMS\ref{}, ALICE\ref{} and LHCb\ref{}. ATLAS and CMS have general purpose, and doing the study aiming the precise measurement of the SM and the discovery of the physics beyond the SM. ALICE is a heavy-ion detector, which studies quark-gluon plasma, while LHCb is a forward-based experiment which is doing the study of heavy-flavour physics. \\
This thesis is based on the data taken by the ATLAS experiment,which is described in the next section.

\section{ATLAS Detector}
\label{sec:detector}
The ATLAS detector is a multi purpose detector, and designed to measure all standard model particles produced by the LHC.
A overview of the whole ATLAS detector is shown in Figure\ref{{fig:CERN}}.

The ATLAS is a cylindrical detector, which is 44~m in length and 25~m in height. It is consists of two areas; the central regions, referred as barrel, and the forward regions, referred as end-caps.

\begin{figure}[tbp]
\begin{center}
%\subfigure[]{
 \includegraphics[width=1.0\textwidth,keepaspectratio]{figures/detector/ATLAS}
%}
\caption{
The whole picture of the ATLAS detector
}
\label{fig:ATLAS}
\end{center}
\end{figure}

\subsection{The ATLAS coordinate system}
The common coordinate system used is right-handed coordinate system with its origin at the interaction point. The z-axis is along the beam, the y-axis is points upwards and the x-axis is points towards the center of the LHC. The azimuthal angle $\phi$ is the angle around the beam pipe, measured in x-y plane with respect to the x-axis. The polar angle $\theta$ is the angle to z-axis. A commonly used variable is the pseudorapidity $\eta$, which is defined as:
\begin{eqnarray*}
\eta &=& -ln[tan(\theta/2)]
\end{eqnarray*}
The small $\eta$ corresponds the the center part of the detector, while high $\eta$ regions points the beam axis. \\
%The other variable used often is the distance in the cylindrical coordinate system:
%$
%\Delta R = \sqrt{\Delta \eta^2 + \Delta \phi^2}
%$

\subsection{Magnet}
The ATLAS magnet system consists of four large superconducting magnets with a dimension of 22 m in diameter and 26 m in length with a stored energy of 1.6 GJ. A solenoid aligned on the beam axis generates 2 T axial magnetic field for the inner detector, which placed inside the calorimeter system. A toroid on the barrel and two toroids on the end-caps are installed and those provide 0.5 and 1T magnetic fields for muon detectors, respectively.

There are different subsystems of the detector, the Inner Detector with the coverage of $|\eta < 2.5|$, the calorimeters with $|\eta < 4.9|$, the muon spectrometer with $|\eta < 2.7|$ from the inside to outside as shown in Figure~\ref{fig:ATLAS}. More details are described below.

\subsection{Inner Detector}
The inner detector (ID) is a tracking detector with a radius of 1.2~m, and 6.2~m in length along the beam pipe. The function of the ID is the reconstruction of the trajectory of charged particles. 
%resolution of measuring the momentum here?
Together with the thin superconducting solenoid magnet producing a magnetic field of 2~T, the track momentum can be measured with the designed resolution of 
\begin{eqnarray*}
    \frac{\sigma_{p_{T}}}{p_{T}} = 0.05~\% ~p_{T} ~[GeV] \oplus 1~\%
\end{eqnarray*}

The ID consists of three subsystems, Pixel, SCT, and TRT, introduced hereafter.
The overview of the ID is shown in Figure~\ref{fig:ID}.

\begin{figure}[tbp]
\begin{center}
\subfigure{
 \includegraphics[width=0.8\textwidth,keepaspectratio]{figures/detector/ID2}
}
\subfigure{
 \includegraphics[width=0.7\textwidth,keepaspectratio]{figures/detector/ID1}
}
\caption{
The whole picture of the ID in section
}
\label{fig:ID}
\end{center}
\end{figure}


\subsubsection{Pixel Detector}
The Pixel Detector is the most inner tracking system. It consists of four cylindrical layers of the silicon (Si) pixel modules and covers a radius of 33.25 to 122.5~mm and $|\eta|$ < 2.5.
%IBL
The innermost layer is called Insertable B-Layer (IBL), which is newly installed before 2015 runs. This layer is lacated at R = 33~mm for ensuring better quality of the track and vertex reconstruction, including the improvement of reconstructing the secondary vertex.
The IBL uses specifically radiation hard pixel sensors with a size of 50~$\mu m \times$ 250$\mu m$ to deal with the increasing luminosity and the radiation damage of the existing systems. 

Surrounding the IBL there are three layers with 1744 pixel sensors with a size of 50~$\mu m \times$ 400$\mu m$. The layer next to the IBL is lacated at R = 50.5~mm. In total there are roughly 80 million readout channels, and the high granularity provides an resolution of 14~$\mu m \times$ 115$\mu m$.

\subsubsection{SCT}
The Semiconductor Tracker(SCT) surrounds the Pixel Detector and located at 30 to 50 ~cm from the beamline. 
It consists of silicon microstrip modules in 4 barrel layers and 9 end-cap disks.
The modules in the barrel have two layer of modules that are slightly rotated each other, which allows for the determination of the position along the strips. There are 2112 modules on the barrel and 1976 modules in the end-cap regions nad the resolution is about 17 $\mu$m $\times$ about 580$\mu$m.

\subsubsection{TRT}
The outside of the SCT there is the Transition Radiation Tracker (TRT). It is lacated 50 to 100~cm from the beamline, covering |η| < 2.0.
It consists of drift tubes filled with gas. The tubes have a diameter of 4~mm and length of 144~m along the beam. Gas is the  mixture of Xenon (70\%), CO2 (27\%), and Oxygen (3\%). 
In addition to measure the momentum, it provides potential for particle identifications. 
The amount of transition radiation, which is emitted by particles crossing the interface of two materials heavily depends on the mass of the crossing particles. Since electrons are the lightest stable charged particles and therefore emit the most transition radiation, the TRT can aid the separation of electrons and pions.

\subsection{Calorimeters}
The calorimeters measure the energy of particles. 

%strategy for measurement
The coverage of the calorimeter extends up to $|\eta|$ = 4.9. A schematic view of the system with the various components can be seen in Figure~\ref{fig:calo}.
It consists of an electromagnetic calorimeter ($|\eta|$ < 3.2) and a hadronic calorimeter ($|\eta|$ < 4.9), covering full $\phi$-range. The details of each electromagnetic and hadronic subcomponents are shown hereafter.

\begin{figure}[tbp]
\begin{center}
%\subfigure[]{
 \includegraphics[width=0.8\textwidth,keepaspectratio]{figures/detector/Calo}
%}
\caption{
The whole picture of the Calorimeter with electromagnetic and hadronic subcomponents
}
\label{fig:calo}
\end{center}
\end{figure}


\subsubsection{ElectroMagnetic Calorimeter}
The ElectroMagnetic Calorimeter (ECAL) measures the electromagnetic showers, initiated by electrons or photons. It is a sampling calorimeter using liquid Argon (LAr) with the lead (Pb) absorbers. They are arranged in an accordion geometry, which enables to cover full $\phi$-range without azimuthal gaps. It is divided into barrel region ($|\eta|$ < 1.4) and end-caps (1.4 < $|\eta|$ < 3.2). The end-caps are divided into a central region (1.375 < $|\eta|$ < 2.5) with finer granularity and a forward region (2.5 < $|\eta|$ < 3.2) with coarser granularity. The design energy resolution $\delta$E/E = 10\% $\sqrt{E/GeV} \oplus$ 0.7\%.

\subsubsection{Hadron Calorimeter}
The Hadronic Calorimeter (HCAL) measures the energy of particles that produce hadronic showers. It is placed outside of the ECAL to detect the hadronic shower made by hadrons.
It is divided into the barrel($|\eta|$ < 1.7), end-caps (1.5 < $|\eta|$ < 3.2) regions. The barrel region is called tile calorimeter and uses scintillating tiles with the steel absorber. The end-caps use LAr as detector material with copper as absorber. 
The design resolution of the hadronic barrel and end caps for jets is $\delta$E/E = 50\% $\sqrt{E/GeV} \oplus$ 3\%.

\subsubsection{Forward Calorimeter}
The Forward Calorimeter (FCAL) measure the energy of both electromagnetic and hadronic particles in the forward region (3.1< $|\eta|$ < 4.9)
It has three layers of absorber material. The first one is copper, optimized for electromagnetic measurements and the other two are made of tungsten for hadronic measurements. 
The design resolution of the forward calorimeter for jets is $\delta$E/E = 100\% $\sqrt{E/GeV} \oplus$ 10\%.

\subsection{Muon Spectrometer}
The Muon Spectrometer (MS) is placed on the most outside of the ATLAS detector to identification the muons and to do precise estimation of their transverse momentum and as well as triggering.
%Muons are the Minimum Ionizing Particles for a large energy range, they cross the calorimeters without being absorbed, and can be measured precisely at MS. 
The coverage of the MS extends up to $|\eta|$ < 2.7. A schematic view of the whole MS is shown in Figure~\ref{fig:MS}.
It is divided into  barrel ($|\eta|$ < 1.4), end caps (1.6 < $|\eta|$ < 2.7) and
transition region (1.4 < $|\eta|$ < 1.6).

\begin{figure}[tbp]
\begin{center}
%\subfigure[]{
 \includegraphics[width=0.8\textwidth,keepaspectratio]{figures/detector/MS}
%}
\caption{
The whole picture of the Muon Spectrometer
}
\label{fig:MS}
\end{center}
\end{figure}

The MS is a ionization detector using gas. When a charged particle crosses the active area, it ionizes the gas, and applied electric field then guides the electrons and ions to be collected. It consists of 4 sub-detectors, described below:

\subsubsection{Monitored Drift Tubes and Cathode Strip Chambers}
The Monitored Drift Tubes (MDTs) perform the precision position measurement of muons 
in $|\eta|$ < 2.7. It consists of the 30~mm diameter drift tube wires filled with gas of Ar and $CO_2$ and wires operated at high voltage to collect the electrons.
The tubes are organized in layers and forming chambers.

The Cathode Strip Chambers (CSCs) perform the same task as the MDTs in the forward region (2 < $|\eta|$ < 2.7), where the counting rate is rather high. 

\subsubsection{Resistive Plate Chambers and Thin Gap Chambers}
The Resistive Plate Chambers (RPCs) are installed in the barrel and providing the muon triggers for the Data Acquisition (DAQ) system.

The Thin Gap Chambers(TGCs) serve the same purpose as the RPCs, but in the forward region. Due to the high event rates, TGCs are designed in a way similar to the CSC.

\subsection{Triggers}
The event rate provided by the LHC is around 40~MHz, which is every 25~ns. It is impossible to store all the information from the detectors, the selection of the event of interest is done with trigger system. The layout of the ATLAS trigger system is shown in Figure~ref{fig:Trigger}.

\begin{figure}[tbp]
\begin{center}
%\subfigure[]{
 \includegraphics[width=1.0\textwidth,keepaspectratio]{figures/detector/Trigger}
%}
\caption{
The layout of the ATLAS trigger system in Run~2
}
\label{fig:Trigger}
\end{center}
\end{figure}

It consists of types of triggers, one is the hardware based Level-1 (L1) trigger, and the other is the software based High Level Trigger (HLT). 

The L1 trigger is implemented with custom electronics and determine the Regions-of-Interest (RoIs), taking as input coarse granularity calorimeter and the muon spectrometer information. These data are treated by sub-systems, the L1Calo and L1Muon, respectively. The reconstructed L1 opjects by the L1Calo and the L1Muon are sent to the L1Topo, and  the selections based on kinematical information is performed. Then all informations from L1Calo, L1Muon, and the L1Topo are sent to the Central Trigger Processor (CTP) and the RoIs are defined. The L1 trigger reduces the event rate from approximately 30~MHz to 100~kHz, and the decision time of the L1 is  2.5~$\mu$s.

The RoIs decided at L1 are then sent to the HLT, which is a farm of large CPU cores.
The HLT reconstructs objets with offine with the full detector information around the RoIs. 
It reduces the rate from the L1 output rate of 100~kHz to approximately 1~kHz on average, within a processing time of about 200~ms.

Events taht pass the trigger selections are recorded on disk, where data can be used for further analysis on off-line.

%upgrades of the detectors?
