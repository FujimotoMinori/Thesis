\chapter{Statistical treatment}

The statistical analysis of the data uses a binned likelihood function constructed as the product of Poisson probability terms,
\begin{equation}     \mathrm{Pois}\,(n|\mu S+B)\left[ \prod_{b\in \text{bins}}^{n} \frac{\mu \nu^{\mathrm{sig}}_{b}+\nu^{\mathrm{bkg}}_{b}}{\mu S+B} \right],
\end{equation} where $\mu$,
a signal strength parameter, multiplies the expected signal yield $\nu^{\mathrm{sig}}_b$ in each histogram bin $b$, and $\nu^{\mathrm{bkg}}_b$ represents the background content for bin $b$.
The dependence of the signal and background predictions on the systematic uncertainties is described by a set of nuisance parameters (NP) $\theta$, which are parameterized by Gaussian or log-normal priors; the latter is used for normalization uncertainties in order to maintain a positive likelihood.
The expected numbers of signal and background events in each bin are functions of $\theta$ and parameterized such that the rates in each category are log-normally distributed for a normally distributed $\theta$.

The priors act to constrain the NPs to their nominal values within their assigned uncertainties.
They are implemented via so-called penalty or auxiliary measurements added to the likelihood which will always increase when any nuisance parameter is shifted from the nominal value.
The likelihood function, $\mathcal{L} (\mu,\theta)$, is therefore a function of $\mu$ and $\theta$.

The nominal fit result in terms of $\mu$ and $\sigma_{\mu}$ is obtained by maximizing the likelihood function with respect to all parameters.
This is referred to as the maximized log-likelihood value, MLL.
The test statistic $q_\mu$ is then constructed according to the profile likelihood: $q_\mu = 2 \ln (\mathcal{L} (\mu, \hat{\hat{\theta_\mu}})/\mathcal{L} (\hat{\mu}, \hat{\theta}))$, where $\hat{\mu}$ and $\hat{\theta}$ are the parameters that maximize the likelihood (with the constraint $0 \leq \hat{\mu} \leq \mu$), and $\hat{\hat{\theta}}_\mu$ are the nuisance parameter values that maximize the likelihood for a given $\mu$.
This test statistic is used to measure the compatibility of the background-only model with the observed data and for exclusion intervals derived with the $CL_s$ method~\cite{Cowan:2010js}.
The limit set on $\mu$ is then translated into a limit on the signal cross section times branching ratio, using the theoretical cross section and branching ratio for the given signal model.



