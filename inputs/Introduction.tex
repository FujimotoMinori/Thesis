\chapter{Introduction}

%\textcolor{red}{This section needs revising}\\
%(1)
The Standard Model (SM) of particle physics is a theory that describes our current understanding of what the building components of matter are and how they interact. The discovery of the Higgs boson in 2012 by ATLAS and CMS experiment confirmed the existence of all particles in the SM~\cite{HIGG-2012-27,CMS-HIG-12-028}. Although it is confirmed that the SM precisely predicts the phenomena up to around the electroweak scale, it is still an incomplete theory, which has problems uncovered such as not explaining the identity of dark matter and the hierarchy problem. It is necessary to explore the possibilities of physics beyond the SM (BSM) through precise measurement of the SM and the search of the new particles.

%(2)
One of the most important topic of research is finding out the underlying mechanism behind the electroweak symmetry breaking, and the precise understanding the electroweak sector of the SM. The electroweak symmetry breaking can be tested in details by investigating the characterization of properties of the Vector Boson Scattering (VBS), since it is sensitive to interactions between the longitudinal components of the gauge bosons.

%(3)
Experimentally, the VBS signature is characterized by the presence of a pair of vector bosons (W, Z, or $\gamma$) and two forward jets with a large separation in rapidity and a large dijet invariant mass.
Both ATLAS and CMS have presented the VBS process~\cite{STDM-2017-19,CMS-SMP-20-001}
%in the same-sign $W^{\pm}W^{\pm}jj$ channel \cite{}, $WZjj$ channel, $ZZjj$ channel 
with the 5 standard deviation in the full-leptonic final state,
%~\footnote{Both of the weak vector boson decay to the leptons}, 
and the 
%ATLAS and CMS have presented results of VBS searches during the Run2 of the LHC \cite{STDM-2017-19,CMS-SMP-20-001} and the 
observation of the VBS process has been claimed, though whole pictures of the VBS measurement including the semi-leptonic final states are needed.

The semi-leptonic channel\footnote{$V(qq')Z(\nu\nu)$, $V(qq')W(l\nu)$ and $V(qq')Z(ll)$ (V = W, Z)}, has some advantages compared to the full-leptonic channel. The hadronically decaying branching fractions are much larger than the leptonically decaying branching fractions, and the use of jet substructure techniques allows good reconstruction efficiency in the high-pT region. The main challenge of the semi-leptonic channel is the presence of large backgrounds from $W + $jets, $Z + $jets and $t\bar{t}$ events. These backgrounds make a VBS measurement in this channel challenging since it is difficult to achieve a favorable signal-to-background ratio. 

%(4)
As the other advantage of the semi-leptonic channel, it is a suitable channel not just for the SM cross-section measurement but also for the interpretation into the BSM physics with more statistics compared to the full-leptonic channel in at high transverse momentum ($p_T$) of the vector bosons and at high invariant mass of the diboson system.
A common way of parameterizing BSM physics in VBS is through an effective field theory (EFT) \cite{Longhitano:1980tm}, which is an model-independent approach avoids having to choose a specific BSM theory. In this EFT framework, VBS can be modified by anomalous quartic gauge couplings (aQGC). 
%%%%%%%%%%%%%%
%On the other hand, an aQGC search is less sensitive to these backgrounds because it is possible to find regions of phase space where the aQGC signal is greatly enhanced over the SM processes, resulting in large signal-to-background ratios. This motivates a search for aQGCs in this channels.
%%%%%%%%%%%%%%
%Searches for aQGCs have been performed by the LEP experiments \cite{delphi1, L3_1, L3_2, opal1, opal2, opal3}, D0 \cite{d0aQGC}, and the LHC experiments both at 8 \TeV
%\cite{STDM-2013-06,STDM-2013-05,STDM-2014-02, STDM-2014-01, STDM-2014-05, STDM-2015-07,STDM-2015-09, CMS-SMP-13-009, CMS-SMP-13-015, CMS-FSQ-13-008}
%and 13 \TeV
%\cite{STDM-2017-23, STDM-2017-24, STDM-2017-20, STDM-2017-06, STDM-2017-26, STDM-2017-19}
%.
The events with high energy bosons are needed to see VBS process, and we will have more statistics in the coming Run and the High-Luminosity LHC (HL-LHC) project. The study in the VBS process will have an important role for searching BSM in coming years with more statistics.
