\chapter{Introduction}
\label{chap:introduction}


%\textcolor{red}{This section needs revising}\\
%(1)
The Standard Model (SM) of particle physics describes the fundamental particles composing matters in the nature and the interactions between them. 
The discovery of a Higgs boson in 2012 by the ATLAS and the CMS experiments confirmed the existence of all particles in the SM~\cite{HIGG-2012-27,CMS-HIG-12-028}. 
Although the SM can precisely predict phenomena up to around the electroweak scale, it is still an incomplete theory model, which has problems uncovered such as the existence of dark matter and the hierarchy problem. 
It is necessary to explore the physics beyond the SM (BSM) through both precise measurements of the SM processes and direct searches of the new particles.

%(2)
One of the most important topics of the research program is finding out the underlying mechanism behind the electroweak symmetry breaking, and the precise understanding of the electroweak sector of the SM. 
The electroweak symmetry breaking can be tested in details by investigating the properties of the Vector Boson Scattering (VBS), since it is sensitive to interactions between the longitudinal components of the gauge bosons.

%
In addition, in case the origin of the electroweak symmetry breaking is not unity or there is a new resonant state at the energy scale that cannot be explored directly at the LHC, the VBS cross section is expected to be modified in particular at the TeV scale.

%(3)
Experimentally, the VBS signature is characterized by the presence of a pair of vector bosons (W, Z, or $\gamma$) and two forward jets with a large separation in rapidity and a large dijet invariant mass.
Both ATLAS and CMS discovered the VBS process~\cite{STDM-2017-19,CMS-SMP-20-001}
%in the same-sign $W^{\pm}W^{\pm}jj$ channel \cite{}, $WZjj$ channel, $ZZjj$ channel 
with the 5 standard deviations from the null hypothesys in the full-leptonic final state, but due to the small branching ratio, the measurement can reach to about 500~GeV.
%~\footnote{Both of the weak vector boson decay to the leptons}, 
To cover wider mass range up to TeV, whole pictures of the VBS measurement including the semi-leptonic final states are needed.

%
This thesis reports the measurememt of the electroweak production of the pair of the vector bosons in association with two hadron jets, including VBS, in the semi-leptonic final states.

The main challenge of the semi-leptonic channel ( $V(qq')Z(\nu\nu)$, $V(qq')W(l\nu)$ and $V(qq')Z(ll)$ (V = W, Z)) is the presence of large backgrounds from $W + $jets, $Z + $jets and $t\bar{t}$ events. 
These backgrounds make a VBS measurement in this channel challenging, but despite of that, the semi-leptonic channel has some advantages compared to the full-leptonic channel. 
The hadronically decaying branching fractions are much larger than the leptonically decaying branching fractions especially in high-$p_T$ region, and the use of jet substructure techniques allows good background rejection despite its high branching ratio. 

%(4)
As the other advantage of the semi-leptonic channel is a suitable channel not only for the SM cross-section measurement but also for the interpretation into the BSM physics with more statistics compared to the full-leptonic channel in at high transverse momentum ($p_T$) of the vector bosons and at high invariant mass of the diboson system.
A common way of parameterizing BSM physics in VBS is through an effective field theory (EFT) \cite{Longhitano:1980tm}, which is an model-independent approach to avoid focussing on a specific BSM theory. In this EFT framework, VBS can be modified by anomalous quartic gauge couplings (aQGC). 


%(5)
This thesis consists of the following contents.
Chapter \ref{chap:theory} summarises briefly the theoretical overview of the SM and VBS analysis, and its interpretation into the EFT studies. 
Chapter \ref{chap:LHCATLAS} describes the experimental setup of the LHC collider and the ATLAS detectors. 
Definition of the physics objects and how they are reconstructed is described in chapter \ref{chap:reconstruction}. 
The signal and the background process definition is listed in chapter \ref{chap:sigbkg}. Chapter \ref{chap:eventselection} shows the event selection applied to the whole analysis.
The background modeling validation and its estimation method is summarized in chapter \ref{chap:modeling}, and the multi-variate techniques used for extracting the signals are explained in chapter \ref{chap:modeling}. The detailed systematic uncertainties are described in chapter \ref{chap:systematics}, the statistical analysis is shown in chapter \ref{chap:statistics}. Finally the results of the measurement is shown in chapter \ref{chap:results}. Chapter \ref{chap:aQGC} describes the interpretation of the results with the EFT. The conclusion of this thesis is summarized in chapter \ref{chap:conclusions}.
