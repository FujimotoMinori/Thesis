\chapter{Introduction}

%%%%%%%%%%%%
\textcolor{red}{This section needs revising}\\
The probe of the electroweak symmetry breaking is one of the most important topics of the LHC experiments.

In the SM, electroweak symmetry breaking (EWSB) is explained by the Brout–Englert–Higgs mechanism.%\cite{}. 
Although many measurements have been made of the properties of the Higgs boson, more information is needed for a complete picture of EWSB. Vector-boson scattering (VBS) is a key probe of EWSB, since it is sensitive to interactions between the longitudinal components of the gauge bosons.
ATLAS and CMS have presented results of VBS searches during the Run2 of the LHC \cite{STDM-2017-19,CMS-SMP-20-001} and the observation of the SM VBS process has been claimed by the ATLAS collaboration.
Experimentally, the VBS signature is characterized by the presence of a pair of vector bosons (W, Z, or $\gamma$) and two forward jets with a large separation in rapidity and a large dijet invariant mass. Previous searches for aQGCs in VBS have focused on channels involving leptonic boson decays ($W(l\nu)$ and $Z(ll)$)~\footnote{Unless otherwise noted, $\ell=e,\mu$.} and photons.
The VBS has already been discovered with the leptonic boson decay with the significance of $\sigma$, though whole pictures of the VBS measurement including the semi-leptonic final states are needed.
The semi-leptonic channels, i.e. $V(qq')Z(\nu\nu)$, $V(qq')W(l\nu)$ and $V(qq')Z(ll)$ (V = W, Z), however, offer some advantages compared to the fully-leptonic channels. The $V(qq')$ branching fractions are much larger than the leptonic branching fractions. In addition, the use of jet substructure techniques allows good reconstruction efficiency in the high-pT region, which is the most sensitive to the beyond standard model physics.
The main challenge of the semi-leptonic channels is the presence of large backgrounds from $W + jets$, $Z + jets$ and $t\bar{t}$ events. These backgrounds make a SM VBS measurement in this channel challenging since it is difficult to achieve a favorable signal-to-background ratio. On the other hand, an aQGC search is less sensitive to these backgrounds because it is possible to find regions of phase space where the aQGC signal is greatly enhanced over the SM processes, resulting in large signal-to-background ratios. This motivates a search for aQGCs in this channels.

A common way of parameterizing BSM physics in VBS is through an effective field theory (EFT) \cite{Longhitano:1980tm}, which is an model-independent approach avoids having to choose a specific BSM theory.
It suits well particularly if the energy scale of the BSM physics is too high for the new resonances of the theory to be observed directly. In this EFT framework, VBS can be modified by anomalous quartic gauge couplings (aQGC). 
%Searches for aQGCs have been performed by the LEP experiments \cite{delphi1, L3_1, L3_2, opal1, opal2, opal3}, D0 \cite{d0aQGC}, and the LHC experiments both at 8 \TeV
%\cite{STDM-2013-06,STDM-2013-05,STDM-2014-02, STDM-2014-01, STDM-2014-05, STDM-2015-07,STDM-2015-09, CMS-SMP-13-009, CMS-SMP-13-015, CMS-FSQ-13-008}
%and 13 \TeV
%\cite{STDM-2017-23, STDM-2017-24, STDM-2017-20, STDM-2017-06, STDM-2017-26, STDM-2017-19}
%.
A typical prediction of aQGCs is an enhancement of the VBS cross-section at high transverse momentum ($p_T$) of the vector bosons and at high invariant mass of the diboson system.
