\chapter{Systematic uncertainties}
In this section, systematic uncertainties considered in the analysis are described. The uncertainties are divided into three categories:
Experimental uncertainties, Background modeling uncertainties, Theoretical uncertainties. Each systematic uncertainty is treated as a nuisance parameter as described in Chapter{}. The systematic variations are estimated on the final discriminant described in Chapter{}.

\section{Experimental uncertainties}

\begin{table}[!hp]
  \centering
  \footnotesize
  \begin{center}
    \begin{tabular}{|l|l|l|}
      \hline
      Source        & Description                     & Analysis Name                                       \\ \hline
      Small-R Jets  &             &  JET\_CR\_JET\_BJES\_Response                            \\
      Small-R Jets  &             &  JET\_CR\_JET\_EffectiveNP\_Detector1                    \\
      Small-R Jets  &             &  JET\_CR\_JET\_EffectiveNP\_Detector2                    \\
      Small-R Jets  &             &  JET\_CR\_JET\_EffectiveNP\_Mixed1                       \\
      Small-R Jets  &             &  JET\_CR\_JET\_EffectiveNP\_Mixed2                       \\
      Small-R Jets  &             &  JET\_CR\_JET\_EffectiveNP\_Mixed3                       \\
      Small-R Jets  &             &  JET\_CR\_JET\_EffectiveNP\_Modelling1                   \\
      Small-R Jets  &             &  JET\_CR\_JET\_EffectiveNP\_Modelling2                   \\
      Small-R Jets  &             &  JET\_CR\_JET\_EffectiveNP\_Modelling3                   \\
      Small-R Jets  &             &  JET\_CR\_JET\_EffectiveNP\_Modelling4                   \\
      Small-R Jets  &             &  JET\_CR\_JET\_EffectiveNP\_Statistical1                 \\
      Small-R Jets  &             &  JET\_CR\_JET\_EffectiveNP\_Statistical2                 \\
      Small-R Jets  &             &  JET\_CR\_JET\_EffectiveNP\_Statistical3                 \\
      Small-R Jets  &             &  JET\_CR\_JET\_EffectiveNP\_Statistical4                 \\
      Small-R Jets  &             &  JET\_CR\_JET\_EffectiveNP\_Statistical5                 \\
      Small-R Jets  &             &  JET\_CR\_JET\_EffectiveNP\_Statistical6                 \\
      Small-R Jets  &             &  JET\_CR\_JET\_Flavor\_Composition                       \\
      Small-R Jets  &             &  JET\_CR\_JET\_Flavor\_Response                          \\
      Small-R Jets  &             &  JET\_CR\_JET\_Pileup\_OffsetMu                          \\
      Small-R Jets  &             &  JET\_CR\_JET\_Pileup\_OffsetNPV                         \\
      Small-R Jets  &             &  JET\_CR\_JET\_Pileup\_PtTerm                            \\
      Small-R Jets  &             &  JET\_CR\_JET\_Pileup\_RhoTopology                       \\
      Small-R Jets  &             &  JET\_CR\_JET\_PunchThrough\_MC16                        \\
      Small-R Jets  &             &  JET\_CR\_JET\_SingleParticle\_HighPt                    \\
      Small-R Jets  &  the scale of forward jets w.r.t. central jets  &  JET\_CR\_JET\_EtaIntercalibration\_TotalStat            \\
      Small-R Jets  &               &  JET\_CR\_JET\_EtaIntercalibration\_Modelling            \\
      Small-R Jets  &             &  JET\_CR\_JET\_EtaIntercalibration\_NonClosure\_highE    \\
      Small-R Jets  &             &  JET\_CR\_JET\_EtaIntercalibration\_NonClosure\_negEta   \\
      Small-R Jets  &             &  JET\_CR\_JET\_EtaIntercalibration\_NonClosure\_posEta   \\
      \hline
      Small-R Jets  & JER                  &  JET\_CR\_JET\_JER\_DataVsMC                 \\
      Small-R Jets  & JER                  &  JET\_CR\_JET\_JER\_EffectiveNP\_1           \\
      Small-R Jets  & JER                  &  JET\_CR\_JET\_JER\_EffectiveNP\_2           \\
      Small-R Jets  & JER                  &  JET\_CR\_JET\_JER\_EffectiveNP\_3           \\
      Small-R Jets  & JER                  &  JET\_CR\_JET\_JER\_EffectiveNP\_4           \\
      Small-R Jets  & JER                  &  JET\_CR\_JET\_JER\_EffectiveNP\_5           \\
      Small-R Jets  & JER                  &  JET\_CR\_JET\_JER\_EffectiveNP\_6           \\
      Small-R Jets  & JER                  &  JET\_CR\_JET\_JER\_EffectiveNP\_7restTerm   \\
      \hline
  \end{tabular}
  \end{center}
\end{table}


\subsection*{Small-$R$ Jet Energy Scale(JES) and Resolution(JER) Uncertainty}
The jet energy scale (JES) uncertainties and jet energy resolution (JER) are measured by calculating the response between MC and data in various kinematic phase space ~\cite{JetUncertainties}.
What is used in this analysis is 30 JES uncertainty and 8 JER uncertainty components.
We also consider the uncertainty on jet vertex tagger (JVT) efficiency~\cite{JVTCalib} and forward jet vertex tagger (fJvt) efficiency.
%We use the configuration \texttt{R4\_CategoryReduction\_SimpleJER.config}



\subsection*{Large-$R$ Jet Energy Scale and Resolution Uncertainty}
\label{sec:fatjetUncert}

The large-$R$ jet energy scale uncertainties are included following the prescription of the
jet substructure group included in the \texttt{JetUncertainties} package~\cite{JSSrecommendation}.
The uncertainty on the \pt scale of jets is evaluated by
comparing the ratio of the jet \pt to track-jet \pt in dijet data and simulation (Rtrk method).
In addition to this ``Baseline'' uncertainty, the uncertainties on track measurements (``Tracking''), differences between Pythia and Sherpa dijet simulations (``Modelling'') and the statistical uncertainty of dijet data (``TotalStat'') are considered.

  \begin{table}[!hp]
  \centering
  \footnotesize
  \begin{center}
    \begin{tabular}{|l|l|l|}
      \hline
      Source        & Description                     & Analysis Name                                       \\ \hline
      Large-R Jets  & \pt scale                       & FATJET\_Medium\_JET\_Rtrk\_Baseline\_pT              \\
      Large-R Jets  & \pt scale                       & FATJET\_Medium\_JET\_Rtrk\_Modelling\_pT             \\
      Large-R Jets  & \pt scale                       & FATJET\_Medium\_JET\_Rtrk\_TotalStat\_pT             \\
      Large-R Jets  & \pt scale                       & FATJET\_Medium\_JET\_Rtrk\_Tracking\_pT              \\

      Large-R Jets  & \pt scale                       & FATJET\_BJT\_JET\_EtaIntercalibration\_Modelling      \\
      Large-R Jets  & \pt scale                       & FATJET\_BJT\_JET\_Flavor\_Composition                 \\
      Large-R Jets  & \pt scale                       & FATJET\_BJT\_JET\_Flavor\_Response                    \\

      Large-R Jets  & Mass resolution                 & FATJET\_JMR                            \\\hline
      Large-R Jets  & JER                             & FATJET\_JER                            \\\hline
\end{tabular}
    \end{center}
  \caption{ Qualitative summary of the systematic uncertainties included in this analysis. }
  \label{tab:syst_summary_sources_3}
  \end{table}

  \clearpage
  
  All other experimental uncertainties are summerized in the Table{}.

\subsection*{Luminosity}
The uncertainty on the integrated luminosity. 
For the 2015+2016 dataset, this uncertainty is 2.1\%, and 2.4\% for the 2017 dataset. The uncertainty for the 2018 data alone is 2.0\%, and the uncertainty for the combined run-2 dataset (2015-2018) is 1.7\% \cite{AtlasLumiRun2}.
The luminosity uncertainty is applied to all MC samples.
\subsection*{Pileup reweighting}
The uncertainty associated with the pileup reweighting is considered\cite{ExtendedPileupReweighting}. To cover the uncertainty on the ratio between the predicted and measure inelastic cross-section in the fiducial volume is covered. The fiducial volume is defined by $M_X > 13\,\GeV$ where $M_X$ is the mass of the non-diffractive hadronic system~\cite{STDM-2015-05}.

 \begin{table}[!hp]
  \centering
  \footnotesize
  \begin{center}
    \begin{tabular}{|l|l|l|l|}
      \hline
      Source             & Description   & Analysis Name    \\ \hline
      Luminosity         & LumiNP        & ATLAS\_LUMI\_2015\_2018 \\ \hline
      Pileup reweighting & PRW\_DATASF   & PRW\_DATASF       \\\hline       
    \end{tabular}
    \end{center}
  \label{tab:syst_summary_sources_1}
 \end{table}



\subsection*{Muons and electrons}
The following systematic uncertainties are applied to electrons and muons in estimations based on the simulation:

\begin{itemize}
\item Identification and reconstruction efficiencies: The efficiencies are measured with the tag and probe method using the $Z$ mass peak.
\item Isolation efficiency: Scale factor and its uncertainty are derived by tag and probe method using the $Z$ mass peak as well.
\item Energy and Momentum scales: These are also measured with $Z$ mass line shape, and provided by the CP groups.
\item Track-to-vertex association efficiency: Only for muons.
\end{itemize}

They are implemented in \texttt{ElectronPhotonFourMomentumCorrection}~\cite{EgammaCalibration},\\
\texttt{ElectronEfficiencyCorrection}~\cite{AsgElectronEfficiencyCorrectionTool}, \\
\texttt{MuonMomentumCorrections} and \texttt{MuonEfficiencyCorrections}~\cite{MCPAnalysisGuidelines}.


\subsection*{Missing transverse energy}
 The missing transverse energy is calculated using physics objects as described in Section~\ref{sec:ObjectDefinition}.
%from the negative vectorial sum of physics objects: muons, electrons, taus, photons, jets and unassociated clusters of calorimeter cells.
As such, all of the systematic errors on the reconstructed components, e.g. the jet energy scale,
result in an uncertainty on $\met$. These are the dominant sources of uncertainty on $\met$.
In addition, there is an uncertainty called the ''Soft Term'', from the unassociated tracks.
The resolution and scale of this soft term are varied within their errors to evaluate their
contribution to the total uncertainty using \texttt{METUtilities}~\cite{METUtilSystematics}.

\begin{table}[!hp]
  \centering
  \footnotesize
  \begin{center}
    \begin{tabular}{|l|l|l|}
      \hline
      Source        & Description                     & Analysis Name                                 \\ \hline
      Electrons     & Energy scale                    &  EG\_SCALE\_ALL                               \\
      Electrons     & Energy resolution               &  EG\_RESOLUTION\_ALL                          \\
      Electrons     & Trigger                         &  EL\_EFF\_Trigger\_TOTAL\_1NPCOR\_PLUS\_UNCOR  \\
      Electrons     & ID efficiency SF                &  EL\_EFF\_ID\_TOTAL\_1NPCOR\_PLUS\_UNCOR      \\
      Electrons     & Isolation efficiency SF         &   EL\_EFF\_Iso\_TOTAL\_1NPCOR\_PLUS\_UNCOR   \\
      Electrons     & Reconstruction efficiency SF    &   EL\_EFF\_Reco\_TOTAL\_1NPCOR\_PLUS\_UNCOR   \\ \hline
      Muons         & \pt\ scale                      &   MUONS\_SCALE                                \\
      Muons         & \pt\ scale (charge dependent)   &   MUON\_SAGITTA\_RHO                    \\
      Muons         & \pt\ scale (charge dependent)          &   MUON\_SAGITTA\_RESBIAS                \\
      Muons         & \pt\ resolution MS               &   MUONS\_MS                                   \\
      Muons         & \pt\ resolution ID               &   MUONS\_ID                                   \\
      Muons         & Isolation efficiency SF         &   MUON\_ISO\_SYS                               \\
      Muons         & Isolation efficiency SF         &   MUON\_ISO\_STAT                               \\
      Muons         & Muon reco \& ID efficiency SF               &   MUONS\_EFF\_STAT                  \\
      Muons         & Muon reco \& ID efficiency SF               &   MUONS\_EFF\_STAT\_LOWPT           \\
      Muons         & Muon reco \& ID efficiency SF               &   MUONS\_EFF\_SYST                  \\
      Muons         & Muon reco \& ID efficiency SF               &   MUONS\_EFF\_SYST\_LOWPT           \\
      Muons         & Track-to-vertex association efficiency SF         &   MUON\_TTVA\_SYS             \\
      Muons         & Track-to-vertex association efficiency SF         &   MUON\_TTVA\_STAT            \\ \hline
      MET           & Soft term                       &   MET\_SoftTrk\_ResoPerp                        \\
      MET           & Soft term                       &   MET\_SoftTrk\_ResoPara                        \\
      MET           & Soft term                       &   MET\_SoftTrk\_Scale                           \\ \hline
      \end{tabular}
      \caption{ Qualitative summary of the systematic uncertainties included in this analysis. }
      \label{tab:syst_summary_sources_2}
    \end{center}
  \end{table}

\subsection{B-tagging uncertainties}
The systematic uncertainties associated to the b-tagging.
They are uncertainties on the scaling factor for taking account for the disagreement of the b-tag efficiency between data and MC. Each separated scale factors and corresponding systematic uncertainties are provided for b-,c-, and light-flavor jets from several measurements.
These b-tagging systematic uncertainties are applied when calculating the SF since they are uncertainties on the SF. b-tagging SF is firstly calculated with all small-R jets (PFlowJets) and applied to the events after having selected non b-tagged tagging jets, same as the uncertainties. 

\begin{table}[!hp]
  \centering
  \footnotesize
  \begin{center}
    \begin{tabular}{|l|l|l|}
      \hline
      Source        & Description                     & Analysis Name  \\ \hline
      B-tagging     & Flavor tagging scale factors    &  FT\_EFF\_Eigen\_B\_0\_AntiKt4PFlowJets                \\
      B-tagging     & Flavor tagging scale factors    &  FT\_EFF\_Eigen\_B\_1\_AntiKt4PFlowJets                \\
      B-tagging     & Flavor tagging scale factors    &  FT\_EFF\_Eigen\_B\_2\_AntiKt4PFlowJets                \\
      B-tagging     & Flavor tagging scale factors    &  FT\_EFF\_Eigen\_C\_0\_AntiKt4PFlowJets                \\
      B-tagging     & Flavor tagging scale factors    &  FT\_EFF\_Eigen\_C\_1\_AntiKt4PFlowJets                \\
      B-tagging     & Flavor tagging scale factors    &  FT\_EFF\_Eigen\_C\_2\_AntiKt4PFlowJets                \\
      B-tagging     & Flavor tagging scale factors    &  FT\_EFF\_Eigen\_C\_3\_AntiKt4PFlowJets                \\
      B-tagging     & Flavor tagging scale factors    &  FT\_EFF\_Eigen\_Light\_0\_AntiKt4PFlowJets            \\
      B-tagging     & Flavor tagging scale factors    &  FT\_EFF\_Eigen\_Light\_1\_AntiKt4PFlowJets            \\
      B-tagging     & Flavor tagging scale factors    &  FT\_EFF\_Eigen\_Light\_2\_AntiKt4PFlowJets            \\
      B-tagging     & Flavor tagging scale factors    &  FT\_EFF\_Eigen\_Light\_3\_AntiKt4PFlowJets            \\
      B-tagging     & Flavor tagging scale factors    &  FT\_EFF\_extrapolation\_AntiKt4PFlowJets              \\
      B-tagging     & Flavor tagging scale factors    &  FT\_EFF\_extrapolation\_from\_charm\_AntiKt4PFlowJets \\ \hline
\end{tabular}
    \end{center}
  \caption{ Qualitative summary of the systematic uncertainties included in this analysis. }
  \label{tab:syst_summary_sources_4}
  \end{table}

\subsection{Quark/Gluon jets uncertainties}
The Flavour response and the Flavour composition uncertainties are derived specifically.
These two uncertainties explains the different jet response to quark and gluon-initiated jets. The flavour response uncertainty is derived by using the dijet events used the alternative samples, which are Phythia~8 and Herwig~$\plus\plus$. The flavour composition uncertainty is usually derived by assuming 50\% quarks and 50\% gluons and takes conservative uncertainty of 100\%. These too conservative uncertainties can limit the sensitivities in topology of this analysis, where quark enriched regions are
expected. In order to reduce these uncertainties, the gluon fraction is studied and these uncertainties are re-derived using specific gluon fraction.

\textcolor{blue}{Put description about how to derive the flavour uncertainties}

\subsection{Track uncertainties}


\subsection{Background Modeling uncertainties}
The systematics uncertainties related to the modeling of the background samples are explained.
By thinking of this, the the difference of the parton shower modeling is considered as the difference of Phythia and Herwig.
\subsection{Reweighting uncertainties}

\section{Theoretical uncertainties}
The systematic uncertainties related to the determination or the calculation of the MC.
They are implemented with EWVVjj signal sample and the background samples that have dominant contributions. 

\subsection{Perturbative QCD matrix-element uncertainties}
\begin{itemize}
\item PDF + $\alpha_s$ uncertainties\\
Since PDF sets are obtained by fitting experimental dataset, there are uncertainties on selecting different type of PDF sets, which originated from the experimental uncertainties from the dataset and from the choice of the functional form adapted in the PDF fits.
The uncertainty from the PDF set choice can be assessed by 2 means. One is by calculating the difference within one PDF set by varying its internal parameters, which is called internal PDF uncertainties. For this first uncertainty the 100 MC replicas of the NNPDF sets are used. The standard deviation of the mean balue of the 100 MC replicas for each bin is taken as the uncertainty. The second is from the difference between the nominal sets of different PDF sets, which is called external PDF uncertainties. This uncertainty is estimated by comparing the nominal distributions with the alternative PDF sets. The uncertainties are combined by taking the envelope of all uncertainties, which means the most varied uncertainty is taken as the final uncertainty. 
$\alpha_s$ uncertainties are from the experimental uncertainties in the determination of it and the truncated fixed order calculations are used for the calculation. This is included to the estimation of the external PDF uncertainties. \textcolor{blue}{?}
\item QCD scale uncertainties\\
QCD scale uncertainties are determined by using different renormalization and factorization scale factor sets. 
The renormalization scale $\mu_R$ is the scale at which the strong coupling $\alpha_s$ is evaluated, while the factorization scale $\mu_F$ is the scale to separate the perturbative part of the hard-interaction from the non-perturbative part of PDF. 
This uncertainties are implemented in the fitting as the NP named \texttt{TheoryQCD\_Z} for Z+jets background sample and \texttt{TheoryQCD\_VBS} for the EW VVjj signal sample. They are estimated by taking the envelope of the 6 set of the combination of the $\mu_R$ and $\mu_F$, which are fixed to be 1, 0.5, or 2.0. The envelop of these sets are used as the uncertainty.
The variation implemented is shown in Appendix~\ref{}. 

\subsection{NLO order EW corrections}
\color{blue}
{The uncertainty related to the corrections of the matrix element when adapted the NLO order calculation. The additional sample is needed to estimate this uncertainty, and this is not currently considered in the analysis.}
\end{itemize}


\section{Normalization}
The list of the normalization of the backgrounds is shown in Table~\ref{tab:prior}.
Some backgrounds are free floated, and others have priors. Since the Diboson background has a similar distribution with the electroweak signal (POI), it is difficult to constraints the normalization of the Diboson as a free floating parameter, therefore we set some prior.

\begin{center}
\begin{tabular}{ |c|c|c|c|c| } 
\hline
 & Background & float or with prior & prior \\
\hline
\multirow{3}{4em}{Merged} & Z & float &       \\ 
& VV                          & prior &   0.5 \\ 
& W                           & prior &   0.1 \\ 
& ttbar                       & prior &   0.1 \\ 
\hline
\multirow{3}{4em}{Resolved} & Z & float &       \\ 
& VV                          & prior &   0.3 \\ 
& W                           & prior &   0.1 \\ 
& ttbar                       & prior &   0.1 \\ 
\hline
\end{tabular}
\caption{\label{tab:prior} The prior setting of the background normalization. }
\end{center}

