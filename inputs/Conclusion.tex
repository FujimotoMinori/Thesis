\chapter{Conclusions}
\label{chap:conclusions}

In this thesis, the vector boson scattering (VBS) in semileptonic final states is studied as the production of a pair of vector bosons associated with a pair of forward jets (EW VV+jj). 

The study is performed using the full Run-2 dataset collected by the ATLAS detector during 2015 to 2018, corresponding to an integrated luminosity of 139~fb$^{-1}$.
The analysis resulted in a measured signal strength of $\mu_{obs} = 0.98^{+ 0.09}_{- 0.09}(\mathrm{Stat})^{+ 0.21}_{- 0.18}(\mathrm{Syst})$.
The background-only hypothesis is rejected with an observed (expected) significance of 5.9 (6.5)~$\sigma$, therefore claiming the observation of the EW VV+jj signal in semileptonic final state. 
Compared to the previous analysis with an integrated luminosity of 35.5~fb$^{-1}$~\cite{STDM-2017-20}, the significance is improved from 2.7 (2.5) to 5.9 (6.5).
Extrapolating the luminosity from 35.5~fb$^{-1}$ to 139~fb$^{-1}$ gives a naive estimation of the significance of about 4~$\sigma$.
The significant improvements to push for more than 5~$\sigma$ can be attributed to several reasons; the updated jet reconstruction using PFlow algorithm, and the updated W/Z boson tagger using 3-variables instead of 2-variables (mass and D2 requirements) previously. 
The multivariate analysis based on the RNN also took a role to have the improvement over the previous round of analysis.
These updates related to the jet reconstruction and selection are validated well in terms of the background modeling and the fitting in this thesis.
The signal strength is used to obtain the fiducial cross section. The fiducial cross section is observed to be $13.3^{+2.99}_{-2.72}$~fb.
%compared with other analyses.
%more precise estimation of the JES/JER uncertainties using the full Run~2 dataset
%good discriminant power of the new RNN 
%The fiducial cross-section of EW VV+jj process is measured to be $\sigma_{EW VV+jj} = * \pm * $~fb.
The measurement of EW VV+jj processes is consistent with the SM prediction, hence the results are interpreted as limits on coefficients of the dimension-8 EFT operators.

The LHC-ATLAS experiment has started acquiring data again after a hiatus of about three years for upgrades. Further data to be acquired, together with data from the planned high-luminosity LHC experiment, will enable a more detailed study and this process will continue to be studied. 
The research methods of this study will serve as a springboard for future research.
The main limitations in the current analysis lie in the large systematic uncertainty related to the V+jets sample mismodeling. This mismodeling is expected to be corrected in the future hence the improvement of the significance by reducing the dedicated uncertainties is expected.
%The final discriminant using RNN also has room of improvement by 

For the SM measurement, now that this process is measured with a significance more than 5~$\sigma$, future directions of analysis could include boson polarization study since longitudinally polarized bosons give more detailed information about the EWSB.
In addition, for the aQGC search, the limits in the EFT signal presented in this study are going to be updated with more statistics in the future. 
Obtained limits should be interpreted in conjunction with the results of other processes, which will be discussed and developed in future studies. Now the limits are obtained one-by-one basis, it is important to study the correlation between the two parameters to get information about what kind of models we need to investigate theoretically, taking the results of the EFT studies.




